\section{Feature Selection}
\label{sec:features}

In the formulation above, we assumed the presence of a feature function
$\featurefn(\statevar, \actionvar)$ that produces a featurized representation of
a state-action pair.  We briefly outline a few basic features that are general
enough to apply to any task in which trajectory transfer is
applicable. \et{``Trajectory transfer'' might not be the right criterion here.}

\begin{itemize}
  \item \textbf{Action bias}: An $|\actionset{}|$-dimensional vector, with each
    component corresponding to a particular action in \actionset{}. Let
    $\indexvar{}_{\actionvar{}}$ be an index associated uniquely with action
    \actionvar{}. The action bias vector is 0 at every component except
    $\indexvar{}_{\actionvar{}}$, where it has a value of 1. This enables us to
    learn whether actions generalize well or poorly and weight them accordingly.
  \item \textbf{Registration cost}: An $(|\actionset{}|+1)$-dimensional vector
    based on the TPS registration cost \regcost{} between \statevar{} and
    $\actionvar{}_{start}$. The registration cost vector consists of a shared
    component, which is always \regcost{}, and an action-specific component, in
    which component $\indexvar{}_{\actionvar{}}$ is set to \regcost{} and every
    other component is left at 0. The shared component allows for a single
    penalty to be applied for any large registration cost. The individual
    components allow for additional adjustments in the cases where actions are
    particularly sensitive to poor registrations
  \item \textbf{Landmarks}: We randomly select a set of ``landmark'' states
    \landmarkset{} from the set of expert demonstrations. The landmark feature
    is an $|\landmarkset{}|$-dimensional vector consisting of the TPS
    registration costs to each of these landmarks. We apply a Gaussian RBF
    kernel to these costs and normalize the vector to sum to 1. This serves to
    identify which portion of the state space we are in via comparison to known
    states, and allows us to prefer states that lie closer to the goal.
\end{itemize}

In our experiments, which we outline in Section~\ref{sec:experiments}, it
suffices to simply concatenate the output of these three feature functions into
a single feature representation. However, depending on the task, one could
very well fold in additional features that rely on domain-specific knowledge.
