\section{Feature Selection}
\label{sec:features}
% Variables used across report
\newcommand{\actionset}{\ensuremath{\mathcal{A}}}
\newcommand{\actionsetsize}{\ensuremath{|\mathcal{A}|}}
\newcommand{\actionvar}{\ensuremath{a}}
\newcommand{\actionsub}[1]{\ensuremath{a_{#1}}}
\newcommand{\actionsubsup}[2]{\ensuremath{a_{#1}^{#2}}}

\newcommand{\demoset}{\ensuremath{\mathcal{D}}}
\newcommand{\demovar}{\ensuremath{d}}
\newcommand{\demosub}[1]{\ensuremath{d_{#1}}}
\newcommand{\demosubsup}[2]{\ensuremath{d_{#1}^{#2}}}

\newcommand{\trajset}{\ensuremath{\mathcal{T}}}

\newcommand{\labelset}{\ensuremath{\mathcal{L}}}
\newcommand{\labelsetsize}{\ensuremath{|\mathcal{L}|}}
\newcommand{\labelvar}{\ensuremath{\ell}}
\newcommand{\labelsub}[1]{\ensuremath{l_{#1}}}
\newcommand{\nsub}[1]{\ensuremath{n_{#1}}}
\newcommand{\sapairsub}[1]{\ensuremath{(s_{#1}, a_{#1})}}

\newcommand{\stateset}{\ensuremath{\mathcal{S}}}
\newcommand{\statevar}{\ensuremath{s}}
\newcommand{\statesub}[1]{\ensuremath{s_{#1}}}
\newcommand{\statesubsup}[2]{\ensuremath{s_{#1}^{#2}}}
\newcommand{\nextstatevar}{\ensuremath{s'}}
\newcommand{\transitionfn}{\ensuremath{T}}
\newcommand{\rewardfn}[1]{\ensuremath{R}}
\newcommand{\goalset}{\ensuremath{\mathcal{G}}}

\newcommand{\policyset}{\ensuremath{{\Pi}}}
\newcommand{\policyvar}{\ensuremath{\pi}}
\newcommand{\policysub}[2]{\ensuremath{\pi_{#1}(#2)}}

\newcommand{\regcost}{\ensuremath{r}}
\newcommand{\indexvar}{\ensuremath{i}}

\newcommand{\landmarkset}{\ensuremath{\mathcal{K}}}

\newcommand{\marginvar}{\ensuremath{m}}
\newcommand{\approxq}{\ensuremath{\tilde{Q}}}
\newcommand{\weights}{\ensuremath{w}}
\newcommand{\weightszero}{\ensuremath{w_0}}
\newcommand{\weightst}{\ensuremath{w^\intercal}}
\newcommand{\featurefn}{\ensuremath{\phi}}
\newcommand{\features}[2]{\ensuremath{\phi({#1}, {#2})}}
\newcommand{\marginslackc}{\ensuremath{C}}
\newcommand{\marginslacksubsup}[2]{\ensuremath{\xi_{#1}^{#2}}}
\newcommand{\bellmanslackc}{\ensuremath{D}}
\newcommand{\bellmanslacksubsup}[2]{\ensuremath{\nu_{#1}^{#2}}}
\newcommand{\bellmanc}{\ensuremath{F}}


In the formulation above, we assumed the presence of a feature function
\featurefn{s}{a} that produces a featurized representation of a state action pair.
We briefly outline a few basic features that are general enough to applied to
any task in which trajectory transfer is applicable. \et{``Trajectory transfer''
  might not be the right criterion here.}

\begin{itemize}
  \item \textbf{Action bias}: An \actionsetsize{}-dimensional vector, with each
    component corresponding to a particular action in \actionset{}. Let $i_a$
    be an index associated uniquely with action \actionvar{}. The action bias vector is 0
    at every component except $i_a$, where it has a value of 1. This enables us
    to learn whether actions generalize well or poorly and weight them
    accordingly.
  \item \textbf{Registration cost}: An $(\actionsetsize{}+1)$-dimensional vector
    based on the TPS registration cost \regcost{} between \statevar{} and \actionsub{start}. The
    registration cost vector consists of a shared component, which is always
    \regcost{}, and an action-specific component, in which component \indexsub{a} is set to
    \regcost{} and every other component is left at 0. The shared component allows for
    a single penalty to be applied for any large registration cost. The
    individual components allow for additional adjustments in the cases where
    actions are particularly sensitive to poor registrations
  \item \textbf{Landmarks}: We randomly select a set of ``landmark'' states \landmarkset{}
    from the set of expert demonstrations. The landmark feature is an
    \landmarksetsize{}-dimensional vector consisting of the TPS registration costs to each of
    these landmarks. We apply a Gaussian RBF kernel to these costs and normalize
    the vector to sum to 1. This serves to identify which portion of the state
    space we are in via comparison to known states, and allows us to prefer
    states that lie closer to the goal.
\end{itemize}

In our experiments, which we outline in Section~\ref{sec:experiments}, it
suffices to simply concatenate the output of these three feature functions into
a single feature representation. However, depending on the task, one could
very well fold in additional features that rely on domain-specific knowledge.
