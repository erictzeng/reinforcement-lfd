\section{Related Work}
Related work for our contribution stems from three areas of research: deformable object manipulation (in particular knot-tieing), max margin policy learning, and \dhm{BLAH}
\subsection{Deformable Object Manipulation}
Our approach can be applied towards a variety of tasks in robotics,
including the manipulation of deformable objects.
In particular, we demonstrate the effectiveness of our approach for
knot tying, a commonly studied manipulation task in robotics.
Previous approaches to knot tying usually depend on rope-specific knowledge
and assumptions.
For instance, in knot planning from observation (KPO), knot theory is used
to recognize rope configurations and define movement primitives in visual
observations of humans tying knots \cite{Morita_ICRA2003, Takamatsu_TransRob2006}.
Existing motion planning approaches for knot tying use topological
representations of rope states (i.e. sequences of rope crossings and their
properties) and define a model for transitioning between topological states
\cite{Saha_ExpRobotics2008, Wakamatsu_IJRR2006}.
Robust open loop execution of knot tying has also been explored \cite{Bell_PhD2010}.

Recently, Schulman et al. used trajectory transfer to enable learning
from human-guided demonstrations of knot tying \cite{Schulmanetal_ISRR2013}.
For a given rope configuration, trajectory transfer is applied to the nearest
demonstration, and the resulting trajectory is executed.
The distance metric used is the thin plate spline (TPS) registration
cost between the rope configurations in the new scene and demonstration.
However, certain demonstrations may be less robust than others: for example,
a demonstration trajectory may involve a grasp that is unnecessarily near
the edge of a rope.
Our approach uses WillSmith to learn a policy that more robustly selects a
demonstration to apply, instead of always selecting the nearest demonstration.
\subsection{Max Margin Policy Learning}

\subsection{Inverse Optimal Control}
\et{I'm not really sure this is a subsection we want, just putting this here for
  now so the next sentence has a home.}  \citet{Dvijotham_ICML2010} also attempt
to directly learn a value function or Q-function for a MDP given sample
transitions generated by an optimal control policy. However, they assume either
a discrete state space or a linear dynamics model. In contrast, our method makes
no assumptions about the size of the state space or the dynamics model, instead
relying on segments of expert demonstrations as a means of navigating the state
space efficiently. \et{Someone who actually understands robotics should probably
  check this statement.}
