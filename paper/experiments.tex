\section{Experiments}
\label{sec:experiments}
% Variables used across report
\newcommand{\actionset}{\ensuremath{\mathcal{A}}}
\newcommand{\actionsetsize}{\ensuremath{|\mathcal{A}|}}
\newcommand{\actionvar}{\ensuremath{a}}
\newcommand{\actionsub}[1]{\ensuremath{a_{#1}}}
\newcommand{\actionsubsup}[2]{\ensuremath{a_{#1}^{#2}}}

\newcommand{\demoset}{\ensuremath{\mathcal{D}}}
\newcommand{\demovar}{\ensuremath{d}}
\newcommand{\demosub}[1]{\ensuremath{d_{#1}}}
\newcommand{\demosubsup}[2]{\ensuremath{d_{#1}^{#2}}}

\newcommand{\trajset}{\ensuremath{\mathcal{T}}}

\newcommand{\labelset}{\ensuremath{\mathcal{L}}}
\newcommand{\labelsetsize}{\ensuremath{|\mathcal{L}|}}
\newcommand{\labelvar}{\ensuremath{\ell}}
\newcommand{\labelsub}[1]{\ensuremath{l_{#1}}}
\newcommand{\nsub}[1]{\ensuremath{n_{#1}}}
\newcommand{\sapairsub}[1]{\ensuremath{(s_{#1}, a_{#1})}}

\newcommand{\stateset}{\ensuremath{\mathcal{S}}}
\newcommand{\statevar}{\ensuremath{s}}
\newcommand{\statesub}[1]{\ensuremath{s_{#1}}}
\newcommand{\statesubsup}[2]{\ensuremath{s_{#1}^{#2}}}
\newcommand{\nextstatevar}{\ensuremath{s'}}
\newcommand{\transitionfn}{\ensuremath{T}}
\newcommand{\rewardfn}[1]{\ensuremath{R}}
\newcommand{\goalset}{\ensuremath{\mathcal{G}}}

\newcommand{\policyset}{\ensuremath{{\Pi}}}
\newcommand{\policyvar}{\ensuremath{\pi}}
\newcommand{\policysub}[2]{\ensuremath{\pi_{#1}(#2)}}

\newcommand{\regcost}{\ensuremath{r}}
\newcommand{\indexvar}{\ensuremath{i}}

\newcommand{\landmarkset}{\ensuremath{\mathcal{K}}}

\newcommand{\marginvar}{\ensuremath{m}}
\newcommand{\approxq}{\ensuremath{\tilde{Q}}}
\newcommand{\weights}{\ensuremath{w}}
\newcommand{\weightszero}{\ensuremath{w_0}}
\newcommand{\weightst}{\ensuremath{w^\intercal}}
\newcommand{\featurefn}{\ensuremath{\phi}}
\newcommand{\features}[2]{\ensuremath{\phi({#1}, {#2})}}
\newcommand{\marginslackc}{\ensuremath{C}}
\newcommand{\marginslacksubsup}[2]{\ensuremath{\xi_{#1}^{#2}}}
\newcommand{\bellmanslackc}{\ensuremath{D}}
\newcommand{\bellmanslacksubsup}[2]{\ensuremath{\nu_{#1}^{#2}}}
\newcommand{\bellmanc}{\ensuremath{F}}


The proposed method for choosing an action was evaluated for the task of knot tying using the Willow Garage PR2 robot platform. 
The demonstrations were collected in the real world, where a human controlled the PR2's gripper to tie a knot.
The labelled examples and the evaluation of the trained policy were run in a simulation environment containing the rope and the PR2.
The rope was simulated as a linked chain of capsules with bending and torsional constraints using Bullet Physics.
To evaluate the trained policy, we measured the success rate of tying an overhand knot performing 5 sequences of actions or less.

The set of demonstrations consist of 148 pairs of point clouds and gripper trajectories.
The point clouds of the ropes were collected using the Asus Xtion Pro and filtered based in color to remove the background.
The gripper trajectories were recorded as a human moved the PR2's gripper to tie a knot.
Even though a human could tie a knot in a sequence of 3 demonstrations, the demonstrations also contained demonstrations that fixed the rope.
\al{should I mention that 93 demonstrations were such that made the rope configuration to be closer to the goal, while the remaining 55 demonstrations were such that made the rope configuration recover from failures?}

The expert labelled examples consist of 1000 pairs of point clouds and actions.
To collect this data, we simulated sequences of actions for various initial point cloud states, with a human in the loop selecting actions.

