\begin{abstract}
We consider the problem of learning from demonstrations
to manipulate deformable objects. Recent
work~\cite{Schulmanetal_IROS2013, Schulmanetal_ISRR2013} has shown
promising results in enabling robotic manipulation of deformable
objects through learning from demonstrations.  Their approach is able
to generalize from a single demonstration to new test situations,
and suggests a nearest
neighbor approach to decide which demonstration to generalize from for
a given test situation.  Such a nearest neighbor approach, however,
ignores important aspects of the problem:  brittleness (versus
robustness) of demonstrations when generalized through this process,
and the extent to which a demonstration makes progress towards the goal.

In this paper, we present a max-margin q-learning-based solution
to the demonstration selection problem that
can account for the variability in robustness of demonstrations and the
sequential nature of our tasks.   We also present experimental
validation of our approach.  We developed a knot-tying benchmark for
evaluating the effectiveness of
our proposed approach.   The
nearest neighbor approach described in \citet{Schulmanetal_ISRR2013} achieves a
68.8\% success rate. Our approach achieves a success rate of 95.2\%.
\end{abstract}
